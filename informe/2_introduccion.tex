\section{Introducción} \label{sec:introduccion}

Las arquitecturas x86 y x64 incluyen un modelo de ejecución llamado SIMD
(\textit{Single Instruction Multiple Data}), que consiste en realizar una
operación sobre un conjunto de datos en una sola instrucción. Este modelo
resulta útil a la hora de procesar contenido multimedia, donde se aplican
algoritmos sobre un conjunto de datos que sigue el mismo formato. A diferencia
del modelo SISD (\textit{Single Instruction Single Data}), este modelo permite
acelerar la ejecución de estos algoritmos paralelizando la manipulación de los
datos de entrada.

Para medir las mejoras de performance que ofrece este modelo, implementaremos 2
Ray Tracers: uno en C y otro en x64 ASM utilizando SIMD. Compararemos tiempos de
ejecución al generar escenas con distintos tipos y cantidades de objetos y
materiales.

La idea de este proyecto como trabajo final fue inspirada en el trabajo de
Martín Mongi \cite{rtmartin}. Además, las implementaciones están basadas en una
implementación anterior en C++ \cite{RayTracerCpp}, que a su vez está basado en
las guías de Shirley \cite{RTIOW}. En la sección \ref{sec:diferencias}
discutiremos que diferencias existen entre este trabajo, el trabajo de Mongi y
la implementación original en C++.
