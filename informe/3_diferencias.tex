%LTeX: language=es-AR
\section{Diferencias con otras implementaciones y trabajos} \label{sec:diferencias}

A lo largo de este trabajo, haremos referencia a las otras implementaciones para
remarcar ciertas diferencias. Es por esto que llamaremos RTMongi, RTC++, RTC y
RTASM a cada una respectivamente.

\subsection{Trabajo de Mongi}

En este trabajo se modifican / agregan las siguientes características:

\paragraph{Objetos} Nuevos objetos, como: planos, cubos y mallas de triángulos
usando archivos \texttt{.obj} en formato STL. Las luces se definen como materiales, y no
como objetos.

\paragraph{Materiales} El color de los objetos no es explícito, sino que se
especifica como un material: lambertiano (color difuso), metálico, dieléctrico, y
un material especial para emitir luz.

\paragraph{Archivo de salida} Se permite parametrizar el \textit{max depth}
(cantidad de veces que un rayo puede rebotar en objetos de la escena) y
\textit{samples per pixel} (cantidad de rayos aleatorios que se mandan dentro
del área correspondiente a cada pixel).

\paragraph{Cámara de escena} Se permite parametrizar la posición del origen, la
dirección en la que apunta, dirección ``arriba'', amplitud del ángulo de
\textit{field of view}, y la apertura de la lente.

\paragraph{Optimización} Nueva estructura KDTree para optimizar el cálculo de
intersecciones entre un rayo y muchos objetos (por ejemplo, en una malla de
triángulos)

\paragraph{Imagen} Para escenas similares, los resultados difieren por la
iluminación.

\subsection{Implementación original en C++}

La implementación en C++ (es decir, RTC++) hace uso de propiedades de la
programación orientada a objetos, como herencia, polimorfismo, etc. Para este
proyecto, se simplificó el código para mejorar la \textit{performance} en la
implementación en ASM (RTASM).

En un principio, lo ideal era simular herencia y polimorfismo con estructuras y
punteros a funciones para facilitar la implementación en C (RTC), pero termino
siendo contraproducente, ya que la diferencia de performance entre C y ASM
resulto ser insignificante. En su lugar, se simplificaron las estructuras de
objetos y materiales para que compartan una estructura \textit{header} común
(herencia) con un campo \texttt{type} que determine el tipo concreto
(polimorfismo).

Por otro lado, como RTC++ hace uso de estructuras de la librería estándar de C++
(\texttt{vector} y \texttt{smart\_ptr}, entre otros), se implementaron
estructuras básicas similares que proveen solamente la funcionalidad que se usa
en este proyecto.

Operaciones como el parsing de la entrada, el overhead inicial de la creación de
objetos y el overhead final en el que se liberan los recursos no se consideran
para medir el rendimiento, y, por lo tanto, se implementan en C.
