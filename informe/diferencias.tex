\section{Diferencias con otras implementaciones y trabajos} \label{sec:diferencias}

\subsection{Trabajo de Mongi}

% diferencias con mongi:
% - objetos: agrega soporte para planos, cubos y mallas de triangulos usando
% archivos .obj en formato stl. No incluye luces como objetos, sino como material
% - materiales: usa material lambertiano para colores fijos, y agrega materiales
% metalicos, y dielectricos. Ademas se agrega fuente de luz como material,
% permitiendo que cualquier objeto de los anteriores sea luz.
% - agrega parametrizacion extra de archivo de salida: max depth, samples per
% pixel
% - agrega parametrizacion de camara: posicion, direccion, vector "arriba",
% amplitud del angulo de field of view, apertura
% - optimizacion de hit para escenas con muchos objetos: KDTree
% - imagen: para escenas similares, los resultados difieren por la forma en la
%   cada uno maneja las luces

% diferencias con implementacion en C++ 
%
% la impl en C++ usa clases y propiedades de herencia, se adaptan usando orden
% del layout en memoria de structs (un puntero a un struct b{ struct a str_a; }
% se puede pasar a una funcion que espera un puntero a struct a). tambien se
% implementaron versiones simplificadas de vector / smart_pointers para manejo
% eficiente de memoria.
% Se separo todo el codigo que se podia optimizar usando simd y se implemento en
% c y luego en asm (c-core y asm-core).

En este trabajo se modifican / agregan las siguientes características:

\paragraph{Objetos} Soporte para dibujar planos, cubos y mallas de triángulos
usando archivos .obj en formato STL. No incluye luces como objetos, se
implementan como materiales.

\paragraph{Materiales} Se usa material lambertiano para dar un color difuso a los
objetos. Además se agregan metales y dieléctricos, y un material especial para
emitir luz difusa.

\paragraph{Archivo de salida} Se permite parametrizar el \textit{max depth}
(cantidad de veces que un rayo puede rebotar en objetos de la escena) y
\textit{samples per pixel} (cantidad de rayos aleatorios que se mandan dentro
del area correspondiente a cada pixel).

\paragraph{Cámara de escena} Se permite parametrizar la posición, dirección,
dirección ``arriba'' (fija la posición y dirección a la que apunta la cámara, la
dirección ``arriba'' indicaría la rotación sobre el eje definido), amplitud del
ángulo de \textit{field of view}, y la apertura.

\paragraph{Optimización} Se agregó una estructura KDTree para optimizar el
método hit de los rayos para escenas con muchos objetos (por ejemplo, malla de
triángulos).

\paragraph{Imagen de salida} Para escenas similares, los resultados difieren por
la forma en la que cada uno maneja las luces.

\subsection{Implementación original en C++}

La implementación en C++ hace uso de propiedades de la programación orientada a
objetos, como herencia, polimorfismo, etc. Esta implementación simula estas
propiedades usando punteros a funciones y propiedades del layout en memoria de
los \texttt{struct}.

Además, hace uso de estructuras de la librería estándar de C++, como
\texttt{vector} y \texttt{smart\_ptr}. Se implementaron estructuras básicas que
proveen solamente la funcionalidad que se usa en este proyecto.

El código tiene una arquitectura distinta, ya que debe permitir la
implementación de algunas partes tanto en C como en ASM. Para esto, movemos las
operaciones que se pueden optimizar con SIMD a un \textit{core}
(\texttt{c\_core.c} o \texttt{asm\_core.s}). Ambos implementan la interfaz
definida en \texttt{core.h}.

Operaciones como el parsing de la entrada y el overhead inicial de la creación de
objetos no se consideran para medir el rendimiento, y por lo tanto, no entran en
algún core y se implementan en C. Por otro lado, los métodos de los objetos de
una escena, o de los materiales de los mismos, si se implementan en cada
core ya que son secciones críticas del programa.
