\newpage
\section{Resultados y Conclusión} \label{sec:conclusion}

En este trabajo reescribimos en C un ray tracer inicialmente hecho en C++, y
luego en ASM haciendo uso de instrucciones SIMD para mejorar la performance del
mismo.

La mejora obtenida (calculada como la relación entre $t_C/t_{ASM}$), ronda entre
5 y 7, es decir, RTASM es entre 5 y 7 veces más rápido que RTC, dependiendo de
la escena que se utilice. Este rango se debe a que nuestra implementación hace
uso de números aleatorios en muchos aspectos de la generación de la imagen, como
la generación de distintos rayos para un mismo pixel, o la reflexión/refracción
de rayos en materiales dieléctricos, entre otros.

Si bien la reescritura de código C en ASM puede resultar difícil en un principio
si no es familiar, es innegable que los beneficios en performance realmente lo
valen. En este caso, obtuvimos un resultado similar al trabajo de Mongi
\cite{rtmartin}, cuya implementación con SIMD fue 6 veces más rápida que la
implementación en C.

Las operaciones que utilizamos en este trabajo simplemente paralelizaron el
cálculo de operaciones básicas entre vectores, como la suma o la resta. También
se hicieron cálculos más complejos, como el producto interno o el producto
vectorial, que simplificaron una operación que suele tomar muchas instrucciones
en una sola. Teniendo en cuenta que nuestra implementación fue realmente una
escritura de otro existente, y no fue hecha desde cero con la intención de
aprovechar a las instrucciones SIMD al máximo, resulta interesante pensar en que
mejora podríamos lograr reescribiendo nuevamente los algoritmos.

% Conclusiones extras sobre como se diferencia con Mongi y como se podría
% mejorar la performance
%
% Una de las principales diferencias con RTMongi, y que también puede ser el
% origen de una nueva mejora en la performance, tiene que ver con la iluminación.
% Por ahora, nuestro Ray Tracer calcula los colores de los materiales haciendo
% ``rebotar'' el rayo hasta que impacte con otro material emisor de luz. Por otro
% lado, RTMongi le asigna un color a cada objeto, y calcula el color final con la
% intensidad, y el ángulo con el que incide la luz sobre la superficie de impacto.
% 
% Podríamos incluir esta lógica en los casos en los que un rayo se pierde en el
% vacío, o cuando se alcanza el \texttt{max\_depth} de la imagen. En ese caso, en
% lugar de asignarle un color negro al rayo (como hacemos ahora), podríamos
% recorrer todos los objetos con materiales luminosos y acumular el aporte que
% realiza cada uno. Esto nos permitiría obtener imágenes con buena iluminación,
% incluso con un \texttt{max\_depth=1}, lo cual por el momento no es posible.
