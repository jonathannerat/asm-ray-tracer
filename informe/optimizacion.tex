\section{Optimización} \label{sec:optimizacion}

Como vimos hasta ahora, la generación de imágenes requiere realizar muchas
operaciones vectoriales. Resulta natural, entonces, intentar optimizarlas para
lograr un impacto positivo en la performance general.

El primer intento consistió en identificar las funciones que realizaban
operaciones vectoriales, y moverlas a un \textit{header} \texttt{core.h} que
sería implementado tanto en C como en ASM. En C se implementaron las funciones
de manera directa:

\begin{lstlisting}[language=C]
typedef struct {
    float x;
    float y;
    float z;
    float _;
} vec3;

vec3 vec3_sum(vec3 a, vec3 b) {
    return (vec3){a.x + b.x, a.y + b.y, a.z + b.z};
}
\end{lstlisting}

, mientras que en ASM, se utilizaron las instrucciones correspondientes:

\begin{lstlisting}[language={[x86masm]Assembler}]
vec3_sum:
    addss xmm1, xmm3
    addps xmm0, xmm2
\end{lstlisting}

Esto se hizo para funciones sobre vectores, como \texttt{vec3\_prod/sub/dot} que
calculaban producto (componente a componente), resta y producto interno, entre
otras. También se hizo para las funciones \textit{hit} de objetos (\textsc{PlaneHit},
\textsc{SphereHit}, etc.), y \textit{scatter} de materiales
(\textsc{LambertianScatter}, \textsc{MetalScatter}, etc).

Cuando se comparó la performance entre las implementaciones, la diferencia era
insignificante. En algunos casos RTASM obtenía un tiempo menor, pero no lo hacia
de forma consistente.

El enfoque de reescribir solamente las funciones y mantener el resto fuera del
\texttt{core} exactamente igual, fue en principio conveniente, pero el overhead
causado por los saltos en el programa por llamadas a funciones posiblemente
anulo todas las ganancias obtenidas.

Además, la convención de llamadas de C establecía que las estructuras con las
que representamos a los vectores (\texttt{vec3}) debían ser pasadas en 2
registros XMM en lugar de 1. Por lo que algunas funciones que se podían resolver
en 1 instrucción pasaban a utilizar 2, o requerían un paso previo para combinar
los 2 registros XMM en uno solo, y otro posterior para volver a separarlo.

Todo esto nos llevó a un segundo enfoque, que fue prácticamente reescribir toda
la lógica del ray tracer en ASM. Esto nos permitió tener un mayor control sobre
como se ejecutaba el código, no solo en el \textit{packing} de estructuras como
\texttt{vec3}, sino también en, por ejemplo, la utilización óptima de registros
para evitar \texttt{push}/\texttt{pop} al stack innecesarios entre llamadas a
funciones.

Las operaciones que antes requerían una llamada a una función
(\texttt{vec3\_sum}), ahora se podían resolver en una sola instrucción
(\texttt{addps}). La mejora es incluso más evidente en operaciones como
\texttt{vec3\_dot}, que calcula el producto interno entre vectores. La
implementación en C puede llegar a tomar 10 instrucciones, que incluyen
movimientos desde y hacia memoria, mientras que en ASM se puede implementar en
una instrucción: \texttt{vdpps xmm0, xmm1, xmm2, 0xF1} es equivalente a
\texttt{float a = vec3\_dot(b, c)}.
