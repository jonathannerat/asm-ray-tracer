\section{Introducción} \label{sec:introduccion}

Las arquitecturas x86 y x64 (así como muchas otras) incluyen un modelo de ejecución llamado SIMD
(\textit{Single Instruction Multiple Data}), que consiste en realizar una operación sobre un cojunto
de datos en una sula instrucción. Este modelo resulta útil a la hora de procesar contenido
multimedia, donde se aplican algoritmos sobre un conjunto de datos que sigue el mismo formato. A
diferencia del modelo SISD (\textit{Single Instruction Single Data}), este modelo permite acelerar
la ejecución de estos algoritmos paralelizando la manipulación de los datos de entrada.

Para medir las mejoras de performance que ofrece este modelo, implementaremos 2 Ray Tracers: uno en
C y otro en x64 ASM utilizando SIMD. Compararemos tiempos de ejecución al generar escenas con
distintos tipos de objetos, materiales, y cantidades.

La idea de este proyecto como trabajo final fue inspirada en el trabajo de Martín Mongi
\cite{rtmartin}. Además, las implementaciones están basadas en una implementación anterior en C++
\cite{RayTracerCpp}, que a su vez está basado en las guías de Shirley \cite{RTIOW}. En la seccion
\ref{sec:diferencias} discutiremos que diferencias existen entre este trabajo, el trabajo de Mongi y
la implementación original en C++.

% El Ray Tracer implementado en este trabajo esta basado en una implementación en C++
% \footnote[1](https://github.com/jonathannerat/ray-tracing-iow-cpp), que a su vez esta basado en las
% guías de Shirley \footnote[2](https://raytracing.github.io).
