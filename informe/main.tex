% !TeX spellcheck = es_AR
\documentclass[10pt, a4paper]{article}


\usepackage[utf8]{inputenc}
\usepackage[spanish]{babel} % separación silábica en castellano

%\usepackage[paper=a4paper, margin=2cm]{geometry} % especifico márgenes manualmente
\usepackage{a4wide} % margenes un poco más anchos

\usepackage{microtype} % saca warnings de underfull boxes
\usepackage{fancyhdr} % encabezado y pie de página
\usepackage{lastpage} % para que muestre la última página en footer

% Comandos para figuras, graficos, tikz, etc.
\usepackage{tikz}
\usepackage{epsfig}
\usepackage{graphicx}
\usepackage{float}
\usepackage{caption}
\usepackage{subcaption}
\usepackage{svg}

% Comandos para teoremas, etc.
\usepackage{amsmath}
\usepackage{amsthm}
\usepackage{amssymb}

\usepackage[
    backend=biber,
    style=numeric,
    sorting=ynt
]{biblatex}
\usepackage{csquotes}
\addbibresource{recursos.bib}
\usepackage{hyperref}
\hypersetup{
    colorlinks=true,
    linkcolor={blue!80!black},
    filecolor=magenta,
    urlcolor=cyan,
    pdftitle={TPF Orga2 2c2021},
    pdfpagemode=UseNone,
    citecolor=blue,
}

\pagestyle{fancy} % Acomodo el encabezado y pie de página.
\lhead{Organización del Computador II}
\setlength{\headheight}{12pt}
\rhead{Segundo Cuatrimestre 2021}
\cfoot{\thepage /\pageref{LastPage}}

\begin{document}

\hfill \includegraphics[scale=0.5]{imgs/logo_dc.jpg}~\\[0.25cm]

\begin{center}
    \textbf{\Large Ray Tracer en C vs. x64 ASM + SIMD}\\[1cm]
    {\large Jonathan Teran Carballo}\\[0.15cm]
    Universidad de Buenos Aires\\[0.15cm]
    \texttt{jonathan.nerat@gmail.com}
\end{center}


\begin{abstract}
    En este trabajo implementaremos un Ray Tracer en C y en x64 ASM utilizando extensiones SIMD para
    comparar la performance de ambos. El mismo es capaz de renderizar mallas de triangulos, utiliza
    KD Trees para optimizar la detección de colisiones, y permite asignar materiales a cada objeto
    de una escena.
\end{abstract}


\hrulefill
\tableofcontents
\clearpage

\input{introduccion}

%LTeX: language=es-AR
\section{Diferencias con otras implementaciones y trabajos} \label{sec:diferencias}

A lo largo de este trabajo, haremos referencia a las otras implementaciones para
remarcar ciertas diferencias. Es por esto que llamaremos RTMongi, RTC++, RTC y
RTASM a cada una respectivamente.

\subsection{Trabajo de Mongi}

En este trabajo se modifican / agregan las siguientes características:

\paragraph{Objetos} Nuevos objetos, como: planos, cubos y mallas de triángulos
usando archivos \texttt{.obj} en formato STL. Las luces se definen como materiales, y no
como objetos.

\paragraph{Materiales} El color de los objetos no es explícito, sino que se
especifica como un material: lambertiano (color difuso), metálico, dieléctrico, y
un material especial para emitir luz.

\paragraph{Archivo de salida} Se permite parametrizar el \textit{max depth}
(cantidad de veces que un rayo puede rebotar en objetos de la escena) y
\textit{samples per pixel} (cantidad de rayos aleatorios que se mandan dentro
del área correspondiente a cada pixel).

\paragraph{Cámara de escena} Se permite parametrizar la posición del origen, la
dirección en la que apunta, dirección ``arriba'', amplitud del ángulo de
\textit{field of view}, y la apertura de la lente.

\paragraph{Optimización} Nueva estructura KDTree para optimizar el cálculo de
intersecciones entre un rayo y muchos objetos (por ejemplo, en una malla de
triángulos)

\paragraph{Imagen} Para escenas similares, los resultados difieren por la
iluminación.

\subsection{Implementación original en C++}

La implementación en C++ (es decir, RTC++) hace uso de propiedades de la
programación orientada a objetos, como herencia, polimorfismo, etc. Para este
proyecto, se simplificó el código para mejorar la \textit{performance} en la
implementación en ASM (RTASM).

En un principio, lo ideal era simular herencia y polimorfismo con estructuras y
punteros a funciones para facilitar la implementación en C (RTC), pero termino
siendo contraproducente, ya que la diferencia de performance entre C y ASM
resulto ser insignificante. En su lugar, se simplificaron las estructuras de
objetos y materiales para que compartan una estructura \textit{header} común
(herencia) con un campo \texttt{type} que determine el tipo concreto
(polimorfismo).

Por otro lado, como RTC++ hace uso de estructuras de la librería estándar de C++
(\texttt{vector} y \texttt{smart\_ptr}, entre otros), se implementaron
estructuras básicas similares que proveen solamente la funcionalidad que se usa
en este proyecto.

Operaciones como el parsing de la entrada, el overhead inicial de la creación de
objetos y el overhead final en el que se liberan los recursos no se consideran
para medir el rendimiento, y, por lo tanto, se implementan en C.


\input{ray_tracing}

\newpage
\section{Resultados y Conclusión} \label{sec:conclusion}

En este trabajo reescribimos en C un ray tracer inicialmente hecho en C++, y
luego en ASM haciendo uso de instrucciones SIMD para mejorar la performance del
mismo.

La mejora obtenida (calculada como la relación entre $t_C/t_{ASM}$), ronda entre
5 y 7, es decir, RTASM es entre 5 y 7 veces más rápido que RTC, dependiendo de
la escena que se utilice. Este rango se debe a que nuestra implementación hace
uso de números aleatorios en muchos aspectos de la generación de la imagen, como
la generación de distintos rayos para un mismo pixel, o la reflexión/refracción
de rayos en materiales dieléctricos, entre otros.

Si bien la reescritura de código C en ASM puede resultar difícil en un principio
si no es familiar, es innegable que los beneficios en performance realmente lo
valen. En este caso, obtuvimos un resultado similar al trabajo de Mongi
\cite{rtmartin}, cuya implementación con SIMD fue 6 veces más rápida que la
implementación en C.

Las operaciones que utilizamos en este trabajo simplemente paralelizaron el
cálculo de operaciones básicas entre vectores, como la suma o la resta. También
se hicieron cálculos más complejos, como el producto interno o el producto
vectorial, que simplificaron una operación que suele tomar muchas instrucciones
en una sola. Teniendo en cuenta que nuestra implementación fue realmente una
escritura de otro existente, y no fue hecha desde cero con la intención de
aprovechar a las instrucciones SIMD al máximo, resulta interesante pensar en que
mejora podríamos lograr reescribiendo nuevamente los algoritmos.

% Conclusiones extras sobre como se diferencia con Mongi y como se podría
% mejorar la performance
%
% Una de las principales diferencias con RTMongi, y que también puede ser el
% origen de una nueva mejora en la performance, tiene que ver con la iluminación.
% Por ahora, nuestro Ray Tracer calcula los colores de los materiales haciendo
% ``rebotar'' el rayo hasta que impacte con otro material emisor de luz. Por otro
% lado, RTMongi le asigna un color a cada objeto, y calcula el color final con la
% intensidad, y el ángulo con el que incide la luz sobre la superficie de impacto.
% 
% Podríamos incluir esta lógica en los casos en los que un rayo se pierde en el
% vacío, o cuando se alcanza el \texttt{max\_depth} de la imagen. En ese caso, en
% lugar de asignarle un color negro al rayo (como hacemos ahora), podríamos
% recorrer todos los objetos con materiales luminosos y acumular el aporte que
% realiza cada uno. Esto nos permitiría obtener imágenes con buena iluminación,
% incluso con un \texttt{max\_depth=1}, lo cual por el momento no es posible.


\printbibliography

\end{document}
