\newpage
\section{Resultados y Conclusión} \label{sec:conclusion}

En este trabajo reescribimos en C un ray tracer inicialmente hecho en C++, y
luego en ASM haciendo uso de instrucciones SIMD para mejorar la performance de
las operaciones vectoriales.

La mejora obtenida (calculada como la relación entre $t_C/t_{ASM}$), ronda entre
5 y 7 veces más rápido. Si bien la mejora obtenida por Mongi es consistente,
mientras que la nuestra se encuentra en un rango, tenemos que tener en cuenta
que nuestra implementación hace uso de aleatoriedad para calcular el color de
cada pixel.

Una de las principales diferencias con la implementación de Mongi, y que también
puede ser el origen de una nueva mejora en la performance, tiene que ver con la
iluminación. Por ahora, nuestro Ray Tracer calcula los colores de los materiales
haciendo rebotar el rayo hasta que impacte con otro material emisor de luz. Por
otro lado, Mongi le asigna un color a cada objeto, y calcula el color final con
la intensidad, y el ángulo con el que incide la luz sobre la superficie de
impacto. Agregar esta mejora, implicaría que podríamos obtener imágenes con
buena iluminación incluso con un \texttt{max\_depth=1}, lo cual por el momento
no es posible.
